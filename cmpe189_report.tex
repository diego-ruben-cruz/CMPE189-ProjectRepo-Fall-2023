\documentclass[conference]{IEEEtran}
\usepackage{cite}
\usepackage{amsmath,amssymb,amsfonts}
\usepackage{algorithmic}
\usepackage{graphicx}
\usepackage{textcomp}
\usepackage{xcolor}
\graphicspath{{./images/}}

\def\BibTeX{{\rm B\kern-.05em{\sc i\kern-.025em b}\kern-.08em
    T\kern-.1667em\lower.7ex\hbox{E}\kern-.125emX}} 
% Preamble for page numbering

\begin{document}

\title{On Wireless Audio: A Brief Literature Review}

\author{\IEEEauthorblockN{Diego Cruz}
    \IEEEauthorblockA{\textit{Dept. of Computer Science} \\
        \textit{San Jose State University}\\
        San Jose, United States \\
        diego.cruz@sjsu.edu}
    \and
    \IEEEauthorblockN{Shiv Shankar Yadav}
    \IEEEauthorblockA{\textit{Dept. of Computer Science} \\
        \textit{San Jose State University}\\
        San Jose, United States \\
        shivshankar.yadav@sjsu.edu}
    \and
    \IEEEauthorblockN{Mahima Agumbe Suresh}
    \IEEEauthorblockA{\textit{Dept. of Computer Engineering} \\
        \textit{San Jose State University}\\
        San Jose, United States \\
        mahima.agumbesuresh@sjsu.edu}
}

\maketitle


\begin{abstract}
    To be added near the end of the research project...
\end{abstract}

\begin{IEEEkeywords}
    Wireless Audio, Fidelity, Wireless Networks
\end{IEEEkeywords}

\section*{Introduction}
Wireless audio is not particularly a new technology, but there have been several different
implementations such as radio waves, Bluetooth audio, and proprietary protocols that
private corporations have made for their respective products.\cite{bhalla_unraveling_2021}

The most well-recognized wireless audio protocol today is the Bluetooth audio protocol
or the IEEE 802.11 WiFi protocol , but there exist different methods and protocols
such as broadcast radio, Apple's native AirPlay, and the Digital Living Network
Alliance (DLNA) standards suite among several others.\cite{parks_wireless_2013}

The intention of this project is to investigate the realm of wireless local area
networks at a small scale. The different protocols and technologies/standards
surrounding wireless audio often are constrained to scale, but some stand out
among others to cover different rooms within a building. However, there will
naturally be a brief review of the history of wireless communications regarding
strictly the transmission and reception of audio signals, or other signals that
are then transcribed to audio.


\section*{Discussion}

The team is open to utilizing diverse resources, such as journal articles, videos and 
documentaries, and empirical records. With this goal in mind, the team aims to locate 
materials that cover the evolution of wireless audio, highlighting the strengths and 
weaknesses of each implementation.

From the preliminary research, it seems that the advent of wireless communications is 
deeply intertwined with the prospect of wireless audio, beginning with Marconi 
popularizing radio communications in the 20th century.\cite{bhalla_unraveling_2021} Deliberate care will have 
to be taken to avoid broadening the discussions of the paper to wireless 
communications in their entirety as opposed to strictly wireless audio. Much more 
research will be required over the course of the semester to gain a comprehensive 
understanding of the subject at hand, and creative measures taken to obtain research 
materials not immediately available from the resources that SJSU offers to students.

A possible avenue of research could involve obtaining specifications from different 
implementations of wireless audio intended for different purposes, such as mobile 
uses, home theater uses, and high-fidelity audio applications. In addition to this, 
different protocols such as Voice over IP (VoIP) may also have to be taken into 
account for the role they may have played in the development of wireless audio 
technologies, even if it is not particularly considered a wireless protocol.

\section*{Conclusion and Future Work}

To be added...

\bibliographystyle{IEEEtran}
\bibliography{cmpe189}

\vspace{12pt}
\end{document}
