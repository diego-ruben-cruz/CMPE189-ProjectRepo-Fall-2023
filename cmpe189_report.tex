\documentclass[conference]{IEEEtran}
\usepackage{cite}
\usepackage{amsmath,amssymb,amsfonts}
\usepackage{algorithmic}
\usepackage{graphicx}
\usepackage{textcomp}
\usepackage{xcolor}
\graphicspath{{./images/}}

\def\BibTeX{{\rm B\kern-.05em{\sc i\kern-.025em b}\kern-.08em
    T\kern-.1667em\lower.7ex\hbox{E}\kern-.125emX}} 
% Preamble for page numbering

\begin{document}

\title{On Wireless Audio: A Brief Literature Review}

\author{\IEEEauthorblockN{Diego Cruz}
    \IEEEauthorblockA{\textit{Dept. of Computer Engineering} \\
        \textit{San Jose State University}\\
        San Jose, United States \\
        diego.cruz@sjsu.edu}
    \and
    \IEEEauthorblockN{Shiv Shankar Yadav}
    \IEEEauthorblockA{\textit{Dept. of Computer Engineering} \\
        \textit{San Jose State University}\\
        San Jose, United States \\
        shivshankar.yadav@sjsu.edu}
    \and
    \IEEEauthorblockN{Mahima Agumbe Suresh}
    \IEEEauthorblockA{\textit{Dept. of Computer Engineering} \\
        \textit{San Jose State University}\\
        San Jose, United States \\
        mahima.agumbesuresh@sjsu.edu}
}

\maketitle


\begin{abstract}
    To be added near the end of the research project...
\end{abstract}

\begin{IEEEkeywords}
    Wireless Audio, Fidelity, Wireless Networks, Bluetooth, 802.11 Wi-Fi
\end{IEEEkeywords}

\section*{Introduction}
Wireless audio is not particularly a new technology, but there have been several different
implementations such as radio waves, Bluetooth audio, and proprietary protocols that
private corporations have made for their respective products.\cite{bhalla_unraveling_2021}
Modern research has included the possibility of using end-to-end audio transmission using laser communications \cite{anthony_approach_2021}, or otherwise creating hybridizations between Wi-Fi and Bluetooth audio implementations to minimize interference.\cite{forenbacher_throughput_2021}

The most well-recognized wireless audio protocol today is the Bluetooth audio protocol
or the IEEE 802.11 WiFi protocol, but there exist different methods and protocols
such as broadcast radio, Apple's native AirPlay, and the Digital Living Network
Alliance (DLNA) standards suite among several others.\cite{parks_wireless_2013}

The intention of this literature review is to investigate the realm of wireless local area
networks at a small scale. The different protocols and technologies/standards
surrounding wireless audio often are constrained to scales built on line of sight,
but some stand out among others to cover different rooms within a building.
However, there will naturally be a brief review of the history of wireless
communications regarding strictly the transmission and reception of audio
signals, or other signals that are then transcribed to audio.

\section*{The History of Wireless Audio}
To understand the history of wireless audio, one must first understand the history of
wireless communications. In general, wireless communications were developed from a need to
communicate across vast distances without the logistical nightmare of cabling across various
environments. A few realworld examples could include having to transmit signals across
mountain ranges, or having to communicate between land and sea, and perhaps between
continents at large.\cite{noauthor_ericsson_2001} Ultimately, the one given the most credit
for inventing modern wireless communication began with Guglielmo Marconi developing the a
telegraph that was capable of utilizing radio frequencies to send morse code messages across
vast distances. His hallmark demonstration was in sending a wireless morse code transmission
from the United Kingdom to Canada.\cite{noauthor_ericsson_2001}

Following innovations in communications hardware such as the triode and the vacuum tube
receiver/transmitter\cite{white_pre-war_2003}, the Marconi telegraph led to the invention of
the radio, resulting in a mass adoption of the radio in the American household.\cite
{noauthor_ericsson_2001} During the 1950s, only then were wireless telephones made more
common with systems like the Ericsson MTA vehicular telephony system. A few decades later,
nearing the end of the Cold War, the GSM specifications established in the 90s
\cite{suresh_introduction_2023}, there were increasingly compact telephones capable of
fitting in the pocket of an individual.\cite{noauthor_ericsson_2001}

It is around this time that Ericsson, a communications company that specialized in telephony,
began to explore avenues of communication surrounding the personal computer, eventually
leading to the creation of the first Bluetooth specification in the late 90s and brought to
market in 2001.\cite{irekvist_bluetooth_2022} Later on, IEEE would modify their 802.11
specification to include Bluetooth audio as well as an wireless audio implementation native
to the Wi-Fi protocol.\cite{noauthor_ieee_2002}

\section*{Discussion}

A few studies have been chosen for the purposes of this particular literature review, to showcase and display different technologies that were used for wireless audio at some point. The aforementioned technologies are listed below:

\begin{itemize}
    \item Radio Frequency (RF)
    \item Infrared (IR)
    \item Bluetooth (BT)
    \item 802.11 Wi-Fi Audio
    \item Near-Field Communication (NFC)
\end{itemize}

In addition to the previously discussed technologies, there are emergent audio technologies that are pushing the envelope of wireless high-fidelity (Hi-Fi) audio. One such technology involves a proprietary.

\section*{Conclusion and Future Work}

To be added...

\bibliographystyle{IEEEtran}
\bibliography{cmpe189}

\vspace{12pt}
\end{document}
